\chapter{Analiza wyników, możliwe drogi dalszych badań}\label{chap:results_analysis}


% Wyniki uzyskane przez zaimplementowany algorytm optymalizacji
% opisane w rozdziałach~\ref{chap:research} oraz~\ref{target_function_chapter} pozwalają

Wyniki przedstawione w rozdziałach~\ref{chap:research} oraz~\ref{target_function_chapter}
wskazują, że zaimplementowany algorytm optymalizacji wytwarza poprawne rozwiązania
dla części problemów. Nawet gdy finalnie wygenerowane brzmienie różni
się od dźwięku docelowego, algorytm wytwarza interesujące barwy dźwięku,
które mogą być wartościowe dla potencjalnych użytkowników.
W niniejszym rozdziale przeprowadzono szczegółową analizę obserwacji,
uzyskanych podczas procesu badawaczego.

\section{Zbiór danych testowych} \label{sec:not_enough_benchmarking_data}

% Automatyczna budowa elektronicznych instrumentów muzycznych
% jest stosunkowo nowym zagadnieniem, więc nie istnieją jeszcze
% takie zbiory danych na których możnaby robić sensowne porównania różnych algorytmów.
% W pracy trzeba było przeszukiwać internet żeby znaleźć jakiekolwiek materiały
% pozwalajace na porównanie. Finalnie też generowanie dźwięku jest zagadnieniem
% sprowadzającym się do subiektywnego odbioru słuchaczy. To czy dana barwa dźwięku
% jest zgodna według jakiejś metryki nie oznacza od razu że dźwięk będzie się podobał
% użytkownikowi. Możliwe że niedoskonałość funkcji celu będzie działała na plus
% dla algorytmu bo pokaże użytkownikowi coś niespodziewanego, a o to właśnie
% chodzi w procesie kreatywnym.

Automatyczna konstrukcja elektronicznych instrumentów muzycznych jest stosunkowo
nowym obszarem badań, w związku z czym brakuje odpowiednich zbiorów danych
umożliwiających porównywanie sprawności różnych algorutmów.
Porównanie wykonane w niniejszej pracy nie stanowi przekroju
przez wiele możliwych typów syntezy, ponieważ w procesie przeglądu
literatury nie udało się znaleźć stosownego zbioru danych. Mimo
to, wykonane porówanie pozwala stwierdzić, że
\textbf{tutaj tekst jak się przekręcą symulacje}.


\section{Optymalizacja parametrów grafu dla problemów o małej złożoności}

Jak wykazały testy przeprowadzone w rozdziale~\ref{target_function_chapter},
zaimplementowany algorytm jest w stanie dokładnie odtworzyć wartości
parametrów grafu dla prostych problemów syntezy
FM~(\ref{fig:param_optimisation_results_spectrograms})
i umiarkowanie złożonych problemów syntezy
\textit{analog modeling}~(\ref{fig:am_param_optimisation_results_spectrograms}).
Testy na małej próbie słuchaczy pozwalają stwierdzić,
że sygnały dźwiękowe wygenerowane dla problemów z rozzdiału~\ref{target_function_chapter}
są nie do rozróżnienia od dźwięków docelowych.

\begin{enumerate}
  \item Nawet jak nie działa to idzie w kierunku celu, często generując interesujące brzmienia po drodze,
  \item Problemy ze zmianami w dynamice, szczególnie pierwsze ułamki sekund,
\end{enumerate}


\section{Potencjał wykorzystania w przemyśle muzycznym}

Wytwarzane przez algorytm barwy są zróżnicowane i brzmią
w sposób ,,muzyczny'' -- nawet gdy wygenerowany dźwięk
znacząco różni się od docelowej barwy, algorytm nie generuje
szumu ani ,,kakofonii''. 
Zaimplementowany algorytm może być uruchomiony na standardowym komputerze,
dostępnym dla przeciętnego użytkownika, co otwiera możliwość jego integracji
z oprogramowaniem typu \textit{digital audio workstation}. Integracja
z oprogramowaniem do produkcji muzyki otwiera drogę do wykorzystania
wtyczek \texttt{VST}, znacznie zwiększając liczbę dostępnych
dla algorytmu węzłów przetwarzania sygnału, potencjalnie usprawniając
jego działanie.


\section{Subiektywna natura porównania}

Generowanie dźwięku stanowi zagadnienie,
które w dużej mierze zależy od subiektywnego odbioru słuchaczy.
Ocena zgodności barwy dźwięku dźwięku według określonej funkcji celu
niekoniecznie przekłada się na subiektywną preferencję użytkownika.
Istnieje możliwość, że niedoskonałości funkcji celu mogą wpływać korzystnie na 
odbiór przez słuchaczy, ponieważ dostarczą nieoczekiwanych efektów, 
które mogą być wykorzystane jako źródło inspiracji w procesie kreatywnym.

\section{Możliwe drogi dalszego rozwoju algorytmu}

\begin{enumerate}
  \item Szybszy wariant DTF,
  \item Różne wielkości okna DTF,
  \item MFCC/Fourier na GPU\@.
  \item Lepsze analizy funkcji celu - brakuje tutaj prac.
  \item Integracja z softwami DAW\@, bo na MCU w środku zwykłego syntha raczej nie wejdzie.
\end{enumerate}


