\chapter{Analiza wyników, możliwe drogi dalszych badań}\label{chap:results_analysis}

\begin{center}
  \large Na razie luźnym językiem
\end{center}

\section{Analiza wyników}

\begin{enumerate}
  \item Fajnie że działa i jest w stanie wygenerować podobne dźwięki, dla prostych przypadków 1:1,
  \item Nawet jak nie działa to idzie w kierunku celu, często generując interesujące brzmienia po drodze,
  \item Problemy ze zmianami w dynamice, szczególnie pierwsze ułamki sekund,
  \item Z jakiegoś powodu algorytm nie~,,dociąga'' parametrów do końca skali.
\end{enumerate}

\section{Możliwe drogi dalszego rozwoju}

\begin{enumerate}
  \item Szybszy wariant DTF,
  \item Różne wielkości okna DTF,
  \item MFCC/Fourier na GPU\@.
  \item Lepsze analizy funkcji celu - brakuje tutaj prac.
  \item Integracja z softwami DAW\@, bo na MCU w środku zwykłego syntha raczej nie wejdzie.
\end{enumerate}

