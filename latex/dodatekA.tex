\chapter{Instrukcja wdrożeniowa}
% Jeśli praca skończyła się wykonaniem jakiegoś oprogramowania, to w dodatku powinna pojawić się instrukcja wdrożeniowa (o tym jak skompilować/zainstalować to oprogramowanie).
% Przydałoby się również krótkie~,,\emph{how to}'' (jak uruchomić system i coś w nim zrobić -- zademonstrowane na jakimś najprostszym przypadku użycia). Można z tego zrobić osobny dodatek.

Algorytm optymalizacyjny i środowisko eksperymentowe zostały zaimplementowane
w systemie Linux, powinny działać również na innych systemach z rodziny UNIX
oraz na platformie Windows.

\section{Wymagane oprogramowanie}

\begin{itemize}
  \item Instalacja języka \texttt{Python} w wersji \texttt{3.10} (w momencie pisania
    pracy biblioteka \texttt{llvmlite}, wymagana przez \texttt{librosa}, wspierała najwyżej
    wersję \texttt{3.10}),
  \item instalacja języka \texttt{Rust} w wersji 1.71 lub wyższej.
    Rekomendowana instalacja z wykorzystaniem \url{https://rustup.rs/},
    toolchain w wersji \texttt{stable},
  \item pakiet \texttt{poetry} (\url{https://python-poetry.org/}) do zarządzania zależnościami
    w języku \texttt{Python}.
  \item pakiet \texttt{graphviz} dostępny za pośrednictwem zmiennej środowiskowej \texttt{\$PATH}.
  \item biblioteka \texttt{libasound2-dev} lub jej odpowiednik dla danego
        systemu operacyjnego (nie wymagana w przypadku systemu Windows).
\end{itemize}

\section{Instalacja}

Środowisko eksperymentowe wykorzystuje
bibliotekę \textit{extension module}~\cite{python_extension_module},
zaimplementowaną w języku \texttt{Rust}, proces instalacji wymaga
zainstalowania bibliotek języka \texttt{Python} \textbf{oraz} skompilowania
\textit{extension module}.
W powłoce systemu operacyjnego należy uruchomić następujące polecenia:

\begin{lstlisting}[language=Bash]
poetry install  # instalacja pakietów dla Pythona
poetry run maturin develop  # skompilowanie extension module w Ruście
\end{lstlisting}

\section{Weryfikacja poprawności działania projektu}

Projekt wyposażony jest w testy jednostkowe, które należy uruchomić w celu
zweryfikowania poprawności instalacji:

\begin{lstlisting}[language=Bash]
cargo test  # Uruchomienie testów jednostkowych dla języka Rust
poetry run pytest .  # Uruchomienie testów dla języka Python
\end{lstlisting}


\section{Praca ze środowiskiem eksperymentowym}

Repozytorium projektu zawiera pliki \texttt{.ipynb}
dla środowiska \textit{Jupyter Lab}, które
jest instalowane jako zależność projektu. Repozytorium
zawiera następujące~,,notebooki'':

\begin{enumerate}
  \item \texttt{demo.ipynb} -- prezentuje podstawowe funkcjonalności projektu: 
    budowę grafów, analizę barwy dźwięku oraz generowanie wykresów.
  \item \texttt{target\_function.ipynb} -- wykonuje porównanie między różnymi
    funkcjami celu, opisane w rozdziale~\ref{target_function_chapter}.
  \item \texttt{prototyping.ipynb} -- zawiera kod dynamicznie budujący strukturę
    grafu DSP na podstawie genotypu oraz wykonuje optymalizację struktury i
    parametrów grafu dla porównania opisanego w rozdziale~\ref{chap:research}.
\end{enumerate}


\subsection{Funkcja celu i detale algorytmu genetycznego}

Dostępne funkcje celu oraz algorytm genetyczny optymalizujący strukturę
i parametry grafu zaimplementowane są w pliku \texttt{luthier/dsp.py}.

\subsection{Implementacja algorytmów syntezy i grafu DSP}

Algorytmy związane z syntezą dźwięku i grafem DSP podzielone są na 3 moduły:

\begin{enumerate}
  \item \texttt{node\_traits/lib.rs} -- definicja interfejsów (\textit{traits} w języku \texttt{Rust}) dla węzłów w grafie DSP,
  \item \texttt{node\_macro/lib.rs} -- makro proceduralne~(\cite{proc_macro}) automatycznie implementujące powtarzalne interfejsy zdefiniowane w \texttt{node\_traits},
  \item \texttt{src/lib.rs} - implementacja węzłów DSP oraz grafu DSP, wykorzystująca wyżej wymienione moduły.
\end{enumerate}

