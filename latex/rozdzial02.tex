\chapter{Definicja problemu}\label{chap:problem_definition}

Jak opisano w zakresie pracy~(\ref{chap:thesis_scope}), w pracy rozwiązywany jest problem
budowy grafu przetwarzania sygnałów oraz opracowanie funkcji celu, porównującej dwa
sygnały pod względem ich barwy. Następnie algorytm generujący graf przetwarzania
sygnałów oraz funkcja porównująca barwy sygnałów są wykorzystane do rozwiązania
problemu optymalizacyjnego. 
Graf przetwarzania sygnałów można opisać jako zbiór połączonych węzłów generujących
i przetwarzających sygnał dźwiękowy. Każdy węzeł opisany jest poprzez:

\begin{multicols}{2}
\begin{enumerate}
  \item zbiór wejść,
  \item zbiór wyjść,
  \item operację matematyczną, wykonywaną na sygnale.
\end{enumerate}


  \begin{figure}[H]
  \centering
  \includegraphics{rys05/example_sine_node.png}
  \caption{Przykładowy węzeł w grafie, generujący sygnał sinusoidalny z możliwością modulacji.}
\end{figure}

\end{multicols}

\noindent
Przykładowo, dla syntezy subtraktywnej powszechnie wykorzystywane są następujące typy węzłów:

\begin{multicols}{2}

\noindent
\textbf{Oscylator} (\textit{VCO}):
\begin{enumerate}
  \item Wejścia:
    \begin{itemize}
      \item częstotliwość,
      \item kształt fali.
    \end{itemize}
  \item Wyjścia:
    \begin{itemize}
      \item wygenerowany sygnał.
    \end{itemize}
\end{enumerate}

\noindent
\textbf{Filtr} (\textit{VCF}):
\begin{enumerate}
  \item Wejścia:
    \begin{itemize}
      \item sygnał wejściowy
      \item częstotliwość odcięcia,
      \item rezonans.
    \end{itemize}
  \item Wyjścia:
    \begin{itemize}
      \item przefiltrowany sygnał.
    \end{itemize}
\end{enumerate}

\end{multicols}


\section{Budowa grafu}

\begin{figure}[H]\label{fig:example_graph_definition_chapter}
    \centering
    \includegraphics[width=0.9\linewidth]{rys05/luthier_simple_analog.png}
    \caption{
      Przykładowy graf DSP\@. Wolne wejścia, które nie są modulowane przez
      źródła sygnału w grafie są optymalizowanymi parametrami.
    }
\end{figure}


Pełny graf przetwarzania można opisać za pomocą zbioru węzłów oraz
listy połączeń między węzłami:

$N = [n_1, n_2, .., n_n]$ - liczba węzłów,

$i_{j} = [ p_1, p_2, \ldots, p_n ]$ -- Zbiór wejść (\textit{inputs}) j-go węzła,

$o_{j} = [ p_1, p_2, \ldots, p_n ]$ -- Zbiór wyjść (\textit{outputs}) j-go węzła,

$f_i(x)$ -- operacja wykonywana na sygnale przez i-ty węzeł. % chtex 8

$C = [ \{ o_{(j, k)}, i_{(l, m)} \}, \ldots ] $: lista połączeń między węzłami, opisujący, które 
$k$-te wyjście $j$-go węzła podłączone jest do którego $m$-go wejścia $l$-go węzła.

Nie wszystkie wejścia w grafie muszą być podłączone do któregoś z wyjść, jak pokazuje
diagram~\ref{fig:example_graph_definition_chapter}.
Wejście, które nie zostało nigdzie podłączone przyjmuje jako wartość wejściową parametr 
liczbowy optymalizowany później w funkcji celu~(\ref{eq:target_function}).
W przypadku schematu~\ref{fig:minilogue_diagram} takimi~,,wolnymi'' wejściami są przykładowo
sygnał określający częstotliwość odcięcia filtru sygnału lub parametry określające parametry
generatora obwiedni~(\textit{EG}).
Genotyp opisujący poszczególny graf przetwarzania sygnału składa się z dwóch części:
\begin{enumerate}
  \item fragment decydujący o strukturze grafu, $S = [s_1, s_2,~\ldots]$,
  \item fragment decydujący o wartości parametrów w wolnych wejściach, $P = [p_2, p_2,~\ldots]$.
\end{enumerate}

\subsection{Struktura grafu}\label{sec:graph_structure_definition}

Różne rodzaje syntezy dźwięku wykorzystują inne struktury grafu przetwarzania
sygnałów~\cite{minilogue_diagram}~\cite{digitone_manual}.
Aby umożliwić dostosowanie grafu przetwarzania sygnałów do wykonywania różnych
rodzajów syntezy, struktura grafu przetwarzania sygnałów jest dynamicznie
modyfikowana przez algorytm optymalizacji. Algorytm generujący określoną
strukturę grafu na podstawie genotypu opisany jest w rozdziale~\ref{chap:solution_algorithm}.
Genotyp odpowiadający za strukturę grafu ma formę krotki liczb rzeczywistych $S$.
Praca definiuje funkcję generującą strukturę grafu $G_s$ z genotypu:
\begin{equation}
  G_s(S) = N, C
  \label{eq:graph_structure_generation_function}
\end{equation}

\subsection{Przypisanie parametrów do~,,wolnych wejść''}\label{sec:graph_params_definition}

Po stworzeniu grafu o danej strukturze $G_s$, druga część genotypu wykorzystywana
jest jako wartości poszczególnych parametrów $P$ dla wolnych wejść w grafie przetwarzania
sygnału.

\begin{equation}
  \forall p_j, \forall i_n \text{ gdzie } i_n \notin C, i_n = p_j
  \label{eq:graph_params_assignment}
\end{equation}

\section{Funkcja celu}

Wykorzystana w pracy funkcja celu $F$, oceniająca, jak sygnał wygenerowany ($\bar{x}$) przez algorytm
jest bliski sygnałowi docelowemu ($x$) przedstawiona jest w następujący sposób:

\begin{equation}
  F(x, \bar{x}) = DTW(MFCC(x),~MFCC(\bar{x}))
  \label{eq:target_function}
\end{equation}

\noindent
Gdzie $MFCC$ oznacza \textit{mel-frequency cepstrum coefficients}, natomiast $DTW$ oznacza algorytm
\textit{dynamic time wrapping}~\cite{mfcc_dtw}. Uzasadnienie wybranej funkcji celu opisane jest z rozdziale~\ref{target_function_chapter}.

\section{Problem optymalizacji}

Sygnał wygenerowany przez graf przetwarzania sygnałów zależy od następujących parametrów:

\begin{equation}
  \bar{x} = G(G_s(S), P)
  \label{eq:graph_generated_signal}
\end{equation}

\noindent
Gdzie $G$ jest funkcją generującą sygnał dźwiękowy dla konkretnej struktury grafu
$G_s$~(\ref{sec:graph_structure_definition},~\ref{eq:graph_structure_generation_function}),
dla wartości parametrów przypisanych z
$P$~(\ref{sec:graph_params_definition},~\ref{eq:graph_params_assignment}).
Dla parametrów $S$ oraz $P$, w pracy rozwiązywany jest następujący problem optymalizacyjny:

\begin{equation}
  \text{Minimize}~F(x,~G(G_s(S), P))
  \label{eq:target_function}
\end{equation}

