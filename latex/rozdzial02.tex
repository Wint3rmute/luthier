\chapter{Definicja problemu}\label{chap:problem_definition}

\begin{multicols}{2}
\noindent
\textbf{Oscylator} (\textit{VCO}):
\begin{enumerate}
  \item Wejścia:
    \begin{itemize}
      \item częstotliwość,
      \item kształt fali.
    \end{itemize}
  \item Wyjścia:
    \begin{itemize}
      \item wygenerowany sygnał.
    \end{itemize}
\end{enumerate}

\noindent
\textbf{Filtr} (\textit{VCF}):
\begin{enumerate}
  \item Wejścia:
    \begin{itemize}
      \item częstotliwość odcięcia,
      \item rezonans.
    \end{itemize}
  \item Wyjścia:
    \begin{itemize}
      \item przefiltrowany sygnał.
    \end{itemize}
\end{enumerate}
\end{multicols}

Graf przetwarzania sygnałów można opisać jako zbiór połączonych węzłów generujących
i przetwarzających sygnał dźwiękowy. Każdy węzeł można opisać poprzez:

\begin{enumerate}
  \item zbiór wejść,
  \item zbiór wyjść,
  \item operację matematyczną, wykonywaną na sygnale.
\end{enumerate}

Pełny graf przetwarzania można opisać za pomocą zbioru węzłów oraz
macierzy połączeń między węzłami:

$N$ - liczba węzłów,

$i_{j} = [ p_1, p_2, \ldots, p_n ]$ -- Zbiór wejść (\textit{inputs}) j-go węzła,

$o_{j} = [ p_1, p_2, \ldots, p_n ]$ -- Zbiór wyjść (\textit{outputs}) j-go węzła,

$f_i(x)$ -- operacja wykonywana na sygnale przez i-ty węzeł. % chtex 8

$C = [ \{ o_{(j, k)}, i_{(l, m)} \}, \ldots ] $: zbiór połączeń między węzłami, opisujący, które 
$k$-te wyjście $j$-go węzła podłączone jest do którego $m$-go wejścia $l$-go węzła.

Nie wszystkie wejścia w grafie muszą być podłączone do któregoś z wyjść.
Wejście, które nie zostało nigdzie podłączone przyjmuje jako wartość parametr 
liczbowy optymalizowany później w funkcji celu~\ref{eq:target_function}.
W przypadku schematu~\ref{fig:minilogue_diagram} takimi~,,wolnymi'' wejściami są przykładowo
sygnał określający częstotliwość odciącia filtru sygnału lub parametry określające parametry
generatora obwiedni~(\textit{EG}).


\subsection{Funkcja celu}

Funkcja celu, oceniająca, jak sygnał wygenerowany ($\bar{x}$) przez algorytm jest bliski sygnałowi docelowemu ($x$)
może zostać przedstawiona w następujący sposób:

\begin{equation}
  F(x, \bar{x}) = q
  \label{eq:target_function}
\end{equation}

\noindent
Gdzie $q$ oznacza liczbę rzeczywistą, która jest tym mniejsza, im bardziej zbliżone do siebie są barwy dźwięku sygnałów $x$ i $\bar{x}$,
przy założeniu że oba sygnały przedstawiają dźwięk o tej samej częstotliwości podstawowej.
