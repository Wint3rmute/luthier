\chapter{Podsumowanie}
\label{chap:podsumowanie}
Podsumowanie jest miejscem, w którym należy zamieścić syntetyczny opis tego, o czym jest dokument. W szczególności w pracach dyplomowych w podsumowaniu powinno znaleźć się jawnie podane stwierdzenie dotyczące stopnia realizacji celu. Czyli powinny pojawić się w niej akapity ze zdaniami typu: ,,Podczas realizacji pracy udało się zrealizować wszystkie postawione cele''. Ponadto powinna pojawić się dyskusja na temat napotkanych przeszkód i sposobów ich pokonania, perspektyw dalszego rozwoju, możliwych zastosowań wyników pracy itp. 

\section{Sekcja poziomu 1}% 
Lorem ipsum dolor sit amet eleifend et, congue arcu. Morbi tellus sit amet, massa. Vivamus est id risus. Sed sit amet, libero. Aenean ac ipsum. Mauris vel lectus. 

Nam id nulla a adipiscing tortor, dictum ut, lobortis urna. Donec non dui. Cras tempus orci ipsum, molestie quis, lacinia varius nunc, rhoncus purus, consectetuer congue risus. 

\subsection{Sekcja poziomu 2}
Lorem ipsum dolor sit amet eleifend et, congue arcu. Morbi tellus sit amet, massa. Vivamus est id risus. Sed sit amet, libero. Aenean ac ipsum. Mauris vel lectus. 
\subsubsection{Sekcja poziomu 3}
Lorem ipsum dolor sit amet eleifend et, congue arcu. Morbi tellus sit amet, massa. Vivamus est id risus. Sed sit amet, libero. Aenean ac ipsum. Mauris vel lectus. 
\paragraph{Paragraf 4}
Lorem ipsum dolor sit amet eleifend et, congue arcu. Morbi tellus sit amet, massa. Vivamus est id risus. Sed sit amet, libero. Aenean ac ipsum. Mauris vel lectus. 
\section{Sekcja poziomu 1}% 
Lorem ipsum dolor sit amet eleifend et, congue arcu. Morbi tellus sit amet, massa. Vivamus est id risus. Sed sit amet, libero. Aenean ac ipsum. Mauris vel lectus. 
%\show\chapter
%\show\section
%\show\subsection

%\showthe\secindent
%\showthe\beforesecskip
%\showthe\aftersecskip
%\showthe\secheadstyle
%\showthe\subsecindent
%\showthe\beforesubsecskip
%\showthe\aftersubsecskip
%\showthe\subseccheadstyle
%\showthe\parskip
