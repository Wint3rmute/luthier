\pdfbookmark[0]{Streszczenie}{streszczenie.1}
%\phantomsection
%\addcontentsline{toc}{chapter}{Streszczenie}
%%% Poniższe zostało niewykorzystane (tj. zrezygnowano z utworzenia nienumerowanego rozdziału na abstrakt)
%%%\begingroup
%%%\setlength\beforechapskip{48pt} % z jakiegoś powodu była maleńka różnica w położeniu nagłówka rozdziału numerowanego i nienumerowanego
%%%\chapter*{\centering Abstrakt}
%%%\endgroup
%%%\label{sec:abstrakt}
%%%Lorem ipsum dolor sit amet eleifend et, congue arcu. Morbi tellus sit amet, massa. Vivamus est id risus. Sed sit amet, libero. Aenean ac ipsum. Mauris vel lectus. 
%%%
%%%Nam id nulla a adipiscing tortor, dictum ut, lobortis urna. Donec non dui. Cras tempus orci ipsum, molestie quis, lacinia varius nunc, rhoncus purus, consectetuer congue risus. 
%\mbox{}\vspace{2cm} % można przesunąć, w zależności od długości streszczenia
\begin{abstract}
Praca prezentuje metodę automatycznej konstrukcji grafu przetwarzania sygnałów dźwiękowych, które wykonują syntezę zadanego przez użytkownika dźwięku.
Wytworzony w ramach pracy algorytm może zostać wykorzystany jako narzędzie w pracy inżynierów dźwięku, podczas tworzenia nowych instrumentów elektronicznych lub efektów
specjalnych. W przeciwieństwie do technik wykorzystujących sieci neuronowe jako narzędzia syntezy, wynikiem działania algorytmu
wytworzonego w ramach pracy jest zrozumiały dla człowieka graf przetwarzania sygnałów, przypominający konwencjonalne konfiguracje
syntezatorów dźwięku wykorzystywane w programach do pracy nad dźwiękiem.

% Streszczenie w języku polskim powinno zmieścić się na połowie strony. Drugą połowę powinno zająć streszczenie w języku angielskim. Zwykle w streszczeniu krótko nawiązuje się do tematu pracy, potem przybliża zawartość pracy oraz osiągnięte wyniki. Czasem w streszczeniu zamieszczana jest notka o wykorzystanym stosie technologii. 
% Styl wypowiedzi powinien być odpowiedni (język formalny, styl akademicki). Należy pamiętać, że w języku polskim i angielskim obowiązują inne zasady. Pomijając użycie czasów do określenia czynności wykonywanych lub planowanych zazwyczaj polski tekst naukowy redagowany jest w trybie dokonanym, w~formie bezosobowej. Natomiast zasady dotyczące manuskryptów w języku angielskim zakładają: głos czynny, forma osobowa. Należy pamiętać, że treść obu abstraktów prawdopodobnie nie będą zgadzać się 1:1.

% W streszczeniu nie stosuje się zaawansowanego formatowania (list wyliczeniowych, wyróżnień, tabel itp.). Można jednak je zredagować w formie kilku (dwóch, może trzech) akapitów.  Pierwszy może przybliżać temat pracy, drugi może dotyczyć przebiegu pracy i jej zawartości. Oczywiście redagowanie pracy jest procesem twórczym i trudno tu narzucać jakieś sztywne reguły. Ujmując rzecz ogólnie, streszczenie powinno umożliwić czytelnikowi zapoznanie się z istotą i zawartością pracy.
\end{abstract}
\mykeywords

{
\selectlanguage{english}
\begin{abstract}
  TODO!
% The abstract in Polish should fit on half a page. The other half should be taken up by an abstract in English. Usually the abstract briefly refers to the thesis topic, then introduces the scope of the work and the results achieved. Sometimes the abstract includes a note about the technology stack used. The writing style should be appropriate (formal language, academic style). Note that Polish and English have different rules. Leaving aside the use of tenses indicating actions performed or planned, the Polish scientific text is generally edited in the accomplished mode and in impersonal form. The rules for English manuscripts are: active voice, personal form. Note that the content of the two abstracts will probably not match 1:1.

% Any advanced formatting (enumeration lists, highlights, tables, etc.) is not allowed here. But the abstract might consist of a couple of paragraphs (two, maybe three). The first can introduce the topic of the paper; the second can be about the course of the work and the thesis' content. Of course, writing is a creative process, and it is difficult to impose any rigid rules here. Generally speaking, the abstract should allow the reader to get an idea of the essence and content of the thesis.
\end{abstract}
\mykeywords
}
