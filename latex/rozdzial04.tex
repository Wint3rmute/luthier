\chapter{Optymalizacja struktury grafu DSP oraz jego parametrów} \label{optimization_problem_chapter}
% \section{Układ pracy dyplomowej}
% Istnieje wiele sposobów organizacji prac dyplomowych. W dokumentacji klasy \texttt{memoir} (\url{http://tug.ctan.org/tex-archive/macros/latex/contrib/memoir/memman.pdf}) zamieszczono zalecenia odnoszące się do edytorskiej strony amerykańskich prac dyplomowych. Na stronie 363 można przeczytać, że część początkowa pracy może zawierać:
% \emph{\setlength\multicolsep{0pt}%
% \begin{multicols}{2}
% \begin{enumerate}[topsep=0pt,labelwidth=1.5em]
% \item Title page
% \item Approval page
% \item Abstract
% \item Dedication (optional)
% \item Acknowledgements (optional)
% \item Table of contents
% \item List of tables (if there are any tables)
% \item List of figures (if there are any figures)
% \item Other lists (e.g., nomenclature, definitions, glossary of terms, etc.)
% \item Preface (optional but must be less than ten pages)
% \item Under special circumstances further sections may be allowed
% \end{enumerate}
% \end{multicols}}
% \noindent
% Po części początkowej powinna pojawić się główna część pracy, złożona z kolejnych rozdziałów. Po niej powinna pojawić się część końcowa, na którą zwykle składają się:
% \emph{\setlength\multicolsep{0pt}%
% \begin{multicols}{2}
% \begin{enumerate}[topsep=0pt,labelwidth=1.5em]
% \item Notes (if you are using endnotes and grouping them at the end)
% \item References (AKA ‘Bibliography’ or ‘Works Cited’)
% \item Appendices
% \item Biographical sketch (optional)
% \end{enumerate}
% \end{multicols}}


% Prace dyplomowe powstające na Wydziale Informatyki i Teleinformatyki Politechniki Wrocławskiej standardowo redaguje się w następującym układzie:
% \emph{\setlength\multicolsep{0pt}%
% \begin{multicols}{2}
% \begin{quote}
% \item Strona tytułowa
% \item Streszczenie 
% \item Strona z dedykacją (opcjonalna)
% \item Spis treści  
% \item Spis rysunków (opcjonalny)
% \item Spis tabel (opcjonalny)
% \item Spis listingów (opcjonalny)
% \item Skróty (wykaz opcjonalny)
% \item Abstrakt 
% \item 1. Wstęp 
% \begin{quote}
% \item 1.1 Wprowadzenie 
% \item 1.2 Cel i zakres pracy 
% \item 1.3 Układ pracy 
% \end{quote}
% \item 2. Kolejny rozdział
% \begin{quote}
% \item 2.1 Sekcja
% \begin{quote}
% \item 2.1.1 Podsekcja
% \begin{quote}
% \item Nienumerowana podpodsekcja
% \begin{quote}
% \item Paragraf
% \end{quote}
% \end{quote}
% \end{quote}
% \end{quote}
% \item $\ldots$
% \item \#. Podsumownie i wnioski
% \item Literatura
% \item A. Dodatek
% \begin{quote}
% \item A.1 Sekcja w dodatku
% \end{quote}
% \item $\ldots$
% \item \$. Instrukcja wdrożeniowa
% \item \$. Zawartość płyty CD/DVD
% \item Indeks rzeczowy (opcjonalny)
% \end{quote}
% \end{multicols}}
% Poniżej zamieszczono opis poszczególnych elementów tego układu.
% \begin{description}[font=\normalfont\itshape,leftmargin=1.em,itemindent=1.em,labelwidth=1.5em]
% \item[Streszczenie] -- syntetyczny opis zawartości pracy, zwykle po pół strony w~języku polskim i języku angielskim (podczas wysyłania pracy do analizy antyplagiatowej należy skopiować tekst z tej sekcji do odpowiedniego pola formularza).
% \item[Spis treści] -- powinien być generowany automatycznie, z podaniem tytułów rozdziałów i podrozdziałów wraz numerami stron. Typ czcionki oraz wielkość liter spisu treści powinny być takie same jak w niniejszym wzorcu. Powinna też być zachowana zadeklarowana głębokość numeracji (patrz ustawienia w kodzie źródłowym szablonu poprzedzone komentarzem \texttt{\% INFO: Deklaracja głębokości numeracji}).
% \item[Spis rysunków, Spis tabel, Spis listingów] -- powinny być generowane automatycznie (podobnie jak \emph{Spis treści}). Elementy te są opcjonalne, nie zawsze trzeba je deklarować. Na przykład robienie osobnego spisu tabel zawierającego jedną lub dwie pozycje specjalnie nie ma sensu (taki spis raziłby pustą przestrzenią na stronie).
% \item[Skróty] -- jeśli w pracy występują liczne skróty, to w tej części należy podać ich rozwinięcia.
% \item[Wstęp] -- pierwszy rozdział, w którego pierwszej sekcji \emph{Wprowadzenie} powinien znaleźć się opis dziedziny, w jakiej osadzona jest praca, oraz wyjaśnienie motywacji do podjęcia tematu. W~drugiej sekcji \emph{Cel i zakres} powinien znaleźć się opis celu oraz zadań do wykonania, zaś w~trzeciej sekcji \emph{Układ pracy} -- opis zawartości kolejnych rozdziałów.

% \item[Rozdział] -- logicznie wyróżniona część pracy. Tytuły kolejnych rozdziałów mogą być różne w~zależności od charakteru pracy. Inne będą dla pracy o charakterze analityczno-przeglądowym, inne dla pracy o charakterze projektowym. Niemniej zawartość i objętość rozdziałów powinny być dobrze wyważone. W pracy dyplomowej inżynierskiej kolejnymi rozdziałami mogą być: \emph{Założenia projektowe}, \emph{Analiza wymagań}, \emph{Szczegóły implementacji}, \emph{Testy}. W pracy dyplomowej magisterskiej mniej więcej jej połowę powinno poświęcić się na część badawczą (analityczną). Zwykle w tego typu pracy zaczyna się od przeglądu literatury (w wyniku czego powstać ma opis bieżących osiągnięć w danej dziedzinie, ang.~\emph{state of the art}). Potem sprecyzować można problemy badawcze, rozwiązywaniem których zajęto się w pracy. Kolejne rozdziały mogą dotyczyć opracowanego rozwiązania, przeprowadzonych eksperymentów i analiz ich wyników itd. Zwykle aspekt praktyczny (inżynierski) w pracy magisterskiej nie jest głównym celem, a raczej środkiem, z pomocą którego dąży się do zrealizowania celu badań (czyli podpiera się nim aspekt badawczy). 
% \item[Podsumowanie] -- w rozdziale tym powinny być zamieszczone: podsumowanie uzyskanych efektów oraz wnioski końcowe wynikające z realizacji celu pracy dyplomowej.
% \item[Literatura] -- wykaz źródeł wykorzystanych w pracy (do każdego źródła musi istnieć odpowiednie cytowanie w tekście). Wykaz ten powinien być generowany automatycznie.
% \item[Dodatek] -- miejsce na informacje dodatkowe, jak: \emph{Instrukcja wdrożeniowa}, \emph{Instrukcja uruchomieniowa}, \emph{Podręcznik użytkownika} itp.
% Osobny dodatek powinien być przeznaczony na opis zawartości dołączonej płyty CD/DVD. Założono, że będzie to zawsze ostatni dodatek.
% \item[Indeks rzeczowy] -- lista referencji do istotnych fraz, do których czytelnik będzie chciał sięgnąć. Indeks powinien być generowany automatycznie. Jego załączanie jest opcjonalne.
% \end{description}

% \section{Styl}
% \label{sec:Styl}
% Pisząc tekst pracy dyplomowej należy zadbać o poprawność językową. Praca powinna być napisana językiem formalnym, w stylu akademickim. W liście wyliczeniowej zamieszczonej poniżej zebrano wybrane reguły odnoszące się do zalecanego stylu wypowiedzi. Lista ta nie jest zamknięta. Może ona być rozszerzona o kolejne zalecenia. 
% (przy okazji proszę zauważyć, że tworząc tę listę wykorzystano mechanizm wyrównania wpisów zależnie od długości najdłuższej etykiety, szczegóły widać w kodzie źródłowym szablonu):
% \begin{enumerate}[labelwidth=\widthof{\ref{last-item}},label=\arabic*.]
% \item Praca dyplomowa powinna być napisana w  formie bezosobowej. Co więcej, używane zwroty dobrze jest pisać w formie zwartej. Czyli zamiast ,,w pracy zostało pokazane ...'' lepiej napisać ,,w pracy pokazano ...''. Wynikowy tekst będzie wtedy krótszy oraz czytelniejszy. Taki styl wypowiedzi przyjęto na uczelniach w naszym kraju, choć w krajach anglosaskich preferuje się redagowanie treści w pierwszej osobie.
% \item W tekście pracy można odwołać się do myśli autora, ale nie w pierwszej osobie, tylko poprzez wyrażenia typu: ,,autor wykazał, że ...''. 
% \item Odwołując się do rysunków i tabel należy używać zwrotów typu: ,,na rysunku ... pokazano ...'', ,,w tabeli ... zamieszczono ...''.  Tabela i rysunek to twory nieżywotne, więc nie mogą ,,pokazywać''. Dlatego też zwroty typu ,,rysunek pokazuje'' uznawane są za niepoprawne.
% \item Praca powinna być napisana językiem formalnym, bez wyrażeń żargonowych (,,sejwowanie'' i ,,downloadowanie''), nieformalnych czy zbyt ozdobnych (,,najznamienitszym przykładem tego niebywałego postępu ...'')
% \item Pisząc pracę należy dbać o poprawność stylistyczną wypowiedzi, a więc: trzeba pamiętać, do czego stosuje się ,,liczba'', a do czego ,,ilość'', zamiast ,,szereg elementów'' powinno się pisać ,,wiele elementów''.
% \item W tekstach dotyczących technologii informacyjnych często dochodzi do zapożyczeń z~języka angielskiego. Zapożyczenia te mocno zakotwiczają się w języku, stając się częścią tzw.\ języka branżowego. I trudno zatrzymać ten proces, szczególnie gdy w~języku polskim brakuje odpowiedników. Jednak gdy takie odpowiedniki istnieją, stosowanie zapożyczeń uważane jest za błąd. Podczas redakcji pracy w języku polskim należy unikać kalek językowych. Przykładem niepoprawnie stosowanej kalki językowej jest termin ,,funkcjonalność''. Zaczęto go używać w odniesieniu do rzeczy policzalnych (mówiąc o~,,wielu funkcjonalnościach'' czy też o ,,wybranej funkcjonalności'') podczas gdy według oficjalnych norm językowych ,,funkcjonalność'' odnosi się do rzeczy, które ,,dobrze spełniający swoją funkcję'' (coś jest ,,funkcjonalne'' jeśli dobrze działa, w przeciwieństwie do czegoś ,,niefunkcjonalnego'', czego nie da się używać). Mówiąc więc o cechach systemów informatycznych powinno się robić odniesienia do ich ,,funkcji'', a nie ,,funkcjonalności''. 
% \item Redagowane zdania nie powinny być zbyt długie, by czytelnik był w stanie je zapamiętać i zrozumieć. Dlatego lepiej podzielić zdanie wielokrotnie złożone na pojedyncze zdania.
% \item Zawartość rozdziałów powinna być dobrze wyważona. Nie wolno więc generować sekcji i podsekcji, które mają zbyt mało tekstu lub znacząco różnią się objętością. Zbyt krótkie podrozdziały można zaobserwować w przykładowym rozdziale~\ref{chap:podsumowanie}.
% \item Niedopuszczalne jest pozostawienie w pracy błędów ortograficznych czy tzw.\ literówek -- można je przecież znaleźć i skorygować automatycznie. \addtocounter{enumi}{1000} 
% \item  Tutaj pokazano, jak można zarządzać marginesem w otoczeniu \texttt{enumerate} (proszę zajrzeć do źródeł latexowego kodu tego rozdziału). \label{last-item}
% \end{enumerate}

% \section{Prowadzenie badań}
% Od prac magisterskich oczekuje się wyraźnie wyróżnionego aspektu badawczego. Dlatego podczas redakcji pracy należy zadbać o wykorzystanie odpowiedniego warsztatu naukowego. Poniżej zamieszczono referencje do wybranych materiałów pomocniczych odnoszących się do tej kwestii. 
% \begin{description}
% \item[Stawianie hipotez badawczych] -- krótkie wprowadzenia do pojęcia hipotezy badawczej i problemów badawczych można znaleźć pod adresami: 
% \url{https://panstatystyk.pl/2022/10/11/ogarniamy_hipotezy_i_pytania_badawcze_w_pracy_magisterskiej/}, \url{https://obliczeniastatystyczne.pl/hipoteza-badawcza/}.
% Godny szczególnego polecenia, gdyż podejmuje się temat hipotez badawczych dla badań eksperymentalnych oraz teoretycznych z punktu widzenia informatyki, jest materiał:
% \url{https://troja.uksw.edu.pl/zasoby/praca-dyplomowa.pdf}

% \item[Opracowywanie uzyskanych wyników] -- podczas prowadzenia badań ważne jest zapewnienie ich obiektywności. Na przykład podczas porównywania działania różnych algorytmów (badań eksperymentalnych) dobrze jest stosować uznane metody statystyczne i metryki. Przykładowe metryki zbierane podczas porównywania algorytmów ewolucyjnych oraz sposoby statystycznego opracowywania uzyskanych wyników znaleźć można w prezentacji Eksperymenty obliczeniowe z algorytmami
% ewolucyjnymi i porównania algorytmów (\url{http://aragorn.pb.bialystok.pl/~wkwedlo/EA9.pdf}).

% Ciekawym materiałem zawierającym ogólne wprowadzenie do testów statystycznych jest: \url{https://panstatystyk.pl/2022/10/27/jaki-test-statystyczny-wybrac-cz-i-czyli-ogarniamy-algorytm-wyboru-testu-statystycznego/}. Warto też spojrzeć na darmowe narzędzie do obliczania parametrów wybranych rozkładów: \texttt{EasyFit} (\url{https://easyfit.soft32.com/}).

% W celu wielokryterialnego porównywania rozwiązań systemów informatycznych można zastosować: \emph{Analityczny Proces Hierarchiczny}, \emph{Wnioskowanie rozmyte}, inne metody \emph{Wielokryterialnego podejmowania decyzji}. Więcej materiałów o statystyczny opracowaniu wyników można znaleźć w literaturze (lub w Internecie) pod hasłem \texttt{testowanie hipotez statystycznych}.
% \end{description}
