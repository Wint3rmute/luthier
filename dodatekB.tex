\chapter{Opis za��czonej p�yty CD/DVD}
\label{chap:opis-plyty}
Tutaj jest miejsce na zamieszczenie opisu zawarto�ci za��czonej p�yty. Opis ten jest redagowany przed za�adowaniem pracy do systemy APD USOS, a wi�c w chwili, gdy nieznana jest jeszcze nazwa, jak� system ten wygeneruje dla za�adowanego pliku. Dlatego te� redaguj�c tre�� tego dodatku dobrze jest stosowa� og�lniki typu: ,,Na p�ycie zamieszczono dokument \texttt{pdf} z niniejszej tekstem pracy'' -- bez wskazywania nazwy tego pliku. 

Dawniej obowi�zywa�a regu�a, by nazywa� dokumenty wed�ug wzorca \texttt{W04\_[nr albumu]\_[rok kalendarzowy]\_[rodzaj pracy]}, gdzie \texttt{rok kalendarzowy} odnosi� si� do roku realizacji kursu ,,Praca dyplomowa'', a nie roku obrony. Przyk�adowo wzorzec nazwy dla pracy dyplomowej in�ynierskiej w konkretnym przypadku wygl�da� tak: \texttt{W04\_123456\_2015\_praca in�ynierska.pdf},  Takie nazwy utrwalane by�y w systemie sk�adania prac dyplomowych. Obecnie dzia�a to ju� inaczej.