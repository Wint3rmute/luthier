\chapter{Zalecenia dotyczące formatowania}
% Większa część niniejszego rozdziału ma charakter informacyjny. Treści w nim zawarte mają uzmysłowić czytelnikowi, jak wiele jest parametrów do ustawienia, by zredagowany dokument wyglądał ,,ładnie''. Parametry te poustawiano w szablonie. Wystarczy z niego skorzystać.

% \section{Rozmiar i układ treści na stronach dokumentu}
% Praca dyplomowa powinna być przygotowana do wydruku na papierze formatu A4 w orientacji pionowej.
% Marginesy na stronach parzystych i nieparzystych powinny być jednakowe i mieć następujące wartości:
% lewy = 25mm, prawy = 25mm, górny = 10mm, dolny = 15mm. Wielkość marginesów w szablonie sterowana jest parametrami przedstawionymi na rysunku~\ref{fig:pageLayout}. Margines dolny powinien być mierzony do linii bazowej tekstu stopki.
% %\begin{figure}[htb]
% 	%\centering
% 	%\includegraphics[width=.7\linewidth]{rys03/pageLayout2}
% 	%\caption{Kontrola marginesów i odstępów elementów na stronie} \label{fig:pageLayout}
% %\end{figure}
% \begin{figure}[htb]
% \setlayoutscale{0.43}
% \currentpage
% \drawparameterstrue
% %\drawpage
% \oddpagelayoutfalse
% \drawstock
% \caption{Układ strony nieparzystej dla dokumentu klasy \texttt{memoir}} \label{fig:pageLayout}
% \end{figure}

% Rzeczywisty układ strony zastosowany w niniejszym dokumencie przedstawiono na rysunku~\ref{fig:currentPageLayout}. Lewy i prawy margines są takie same, więc strony parzyste i nieparzyste wyglądają podobnie, z dokładnością do umiejscowienia notatek marginesowych. Taki rezultat zapewniło zastosowanie poniższych komend. 
% \begin{figure}[t]
% \setlayoutscale{0.43}
% \currentstock
% \oddpagelayouttrue
% \twocolumnlayoutfalse
% \drawmarginparstrue
% \drawparametersfalse
% \drawstock
% \caption{Rzeczywisty układ strony nieparzystej w tym dokumencie} \label{fig:currentPageLayout}
% \end{figure}

% \begin{lstlisting}[basicstyle=\footnotesize\ttfamily]
% \setlength{\headsep}{10pt} 
% \setlength{\headheight}{13.6pt} 
% \setlength{\footskip}{\headsep+\headheight}
% \setlength{\uppermargin}{\headheight+\headsep+1cm}
% \setlength{\textheight}{\paperheight-\uppermargin-\footskip-1.5cm}
% \setlength{\textwidth}{\paperwidth-5cm}
% \setlength{\spinemargin}{2.5cm}
% \setlength{\foremargin}{2.5cm}
% \setlength{\marginparsep}{2mm}
% \setlength{\marginparwidth}{2.3mm}
% \checkandfixthelayout[fixed] 
% \linespread{1}
% \setlength{\parindent}{14.5pt}
% \end{lstlisting}


% \section{Strona tytułowa}
% Stronę tytułową przygotowano starając się wypełnić zalecenia zamieszczone na stronie Wydziału Informatyki i Teleinformatyki (\url{https://wit.pwr.edu.pl/studenci/dyplomanci/praca-dyplomowa}, [dostęp dnia 16.12.2022]). Zastosowano więc pakiet \texttt{ebgaramond}. Dostarcza on klonu wymaganej czcionki \texttt{garamond}, jednak bez kształtu \texttt{slanted} i z pewnymi brakami. Na przykład zamiast literki ,,ł'' w zbiorze \texttt{EBGaramond08 Italic} renderuje się samo ,,l'' (braku tego nie ma zbiór \texttt{EBGaramond12}).  Zaletą pakietu  w porównaniu do innych jest to, że generalnie dobrze obsługiwane są w nim polskie znaki oraz że pakiet ten można znaleźć w różnych dystrybucjach latexa (\texttt{MikTeX} instaluje go automatycznie).

% Wielkości znaków użytych do wypełnienia strony tytułowej treścią są następujące:
% \begin{lstlisting}[basicstyle=\footnotesize\ttfamily]
% Politechnika Wrocławska (Garamond Bold 20pt 22pt)
% Wydział Informatyki i Teleinformatyki (Garamond Bold 16pt 18pt)
% Kierunek: Jakiś kierunek (Garamond 14pt 16pt, Garamond Bold 14pt 16pt)
% Specjalność: Jakaś specjalność (Garamond 14pt 16pt, Garamond Bold 14pt 16pt)
% {P}RACA {D}YPLOMOWA (Garamond 26pt 28pt, Garamond 24pt 26pt)
% {I}NŻYNIERSKA (Garamond 26pt 28pt, Garamond 24pt 26pt)
% Tytuł pracy w języku polskim (Garamond Bold 18pt 20pt)
% Title in English (Garamond Bold 18pt 20pt)
% % AUTOR: (Garamond 16pt 18pt) - zamarkowane
% Imię Nazwisko (Garamond Bold 16pt 18pt)
% Opiekun pracy (Garamond 14pt 16pt)
% tytuł/stopień naukowy, Imię Nazwisko (Garamond 14pt 16pt)
% %OCENA PRACY: (Garamond 16pt 18pt) - zamarkowane
% WROCŁAW, 2021 (Garamond 14pt 16pt)
% \end{lstlisting}


% \section{Główny tekst}
% Główny tekst pracy powinien być zredagowany z wykorzystaniem czcionki \texttt{Times}, typ normalny, o wysokości 12pt, z odstępem między liniami równym 14.5pt. Istnieje możliwość zmiany odstępu między liniami za pomocą komendy \verb?\linespread?, jednak zaleca się pozostawienie tego odstępu jak w niniejszym dokumencie (\verb?\linespread{1}?). Wymagania odnośnie kroju pisma pozostałych elementów (nagłówków, stopek itp.) zamieszczono w tabeli~\ref{tab:secfonts}.

% W szablonie zastosowano czcionkę \texttt{texgyre-termes} (dostarcza ją pakiet \texttt{tgtermes}). Czcionka ta jest klonem czcionki \texttt{Times}, w którym obsługiwane jest środkowoeuropejskie kodowanie znaków (podobnie jak w przypadku czcionki \texttt{ebgaramond}, dzięki czemu polskie literki nie są zlepkami dwóch znaków lecz pojedynczymi znakami).  

% Wszelkie przykłady źródeł kodu (fragmenty programów, komendy linii poleceń), nazwy plików i uruchamianych programów powinny być pisane czcionką maszynową. W szablonie czcionką maszynową jest \texttt{t1xtt}. Czcionka ta obsługuje polskie znaki. Dostarcza ją pakiet \texttt{txfonts}, który należy wcześniej zainstalować (MiKTeX zainstaluje go automatycznie podczas pierwszej kompilacji szablonu).   

% % https://en.wikibooks.org/wiki/LaTeX/Lengths
% % there are two different point sizes.  A pdf point is 1/72 inch.  A LaTeX point is 1/72.27 inch.  Thus,
% % the LaTeX point is slightly smaller than a pdf point. 

% %\texttt{baselineskip}: \printlength{\baselineskip}\\
% %\texttt{beforesecskip}: \printlength{\beforesecskip}\\
% %\texttt{aftersecskip}: \printlength{\aftersecskip}\\
% %\texttt{topskip}: \printlength{\topskip}\\
% %\texttt{fontsize}: \showFontSize\\

% \begin{table}[htb]
% \centering
% \caption{Zestawienie czcionek elementów podziału dokumentu, tekstu wiodącego, nagłówka i stopki oraz podpisów (Rozm. -- rozmiar czcionki, Odst. -- \texttt{baselineskip})}
% \label{tab:secfonts}\small
% \begin{tabularx}{\linewidth}{|ll@{\hskip 5pt}l@{\hskip 5pt}lX|} \hline
% Element & Przykład & Czcionka & Rozm. & Odst. \\ \hline\hline
% Nr rozdziału & {\huge\bfseries Rozdział 1 } & \verb?\huge\bfseries? & 25pt & 30pt \\
% Tytuł rozdziału & {\Huge\bfseries Wstęp } & \verb?\Huge\bfseries? & 30pt & 37pt\\
% Nr i tytuł sekcji & {\Large\bfseries 1.1. Wprowadzenie } & \verb?\Large\bfseries? & 17pt & 22pt \\
% Nr i tytuł podsekcji & {\large\bfseries 1.1.1. Cel szczegółowy } & \verb?\large\bfseries? &14.5pt & 18pt\\
% Tytuł podpodsekcji  & {\normalsize\bfseries Założenia } & \verb?\normalsize\bfseries? & 12pt & 14.5pt\\
% Tytuł paragrafu & {\normalsize\bfseries  Podstawy } Opis ... &  \verb?\normalsize\bfseries? & 12pt & 14.5pt\\
% Tekst wiodący & {\normalsize Niniejszy dokument ... } & \verb?\normalsize? & 12pt & 14.5pt\\
% Nagłówek strony & {\small\itshape 3.2. Czcionka wiodąca ...} & \verb?\small\itshape? & 11pt & 13.6pt \\
% Stopka strony & {\small Imię Nazwisko: ...} & \verb?\small? & 11pt & 13.6pt\\
% Podpisy tabel & {\small Tab.~3.1: Zestawienie ...} & \verb?\small? & 11pt & 13.6pt \\
% Podpisy rysunków & {\small Rys.~3.1: Oficjalny ...} & \verb?\small? & 11pt & 13.6pt\\\hline
% \end{tabularx}
% \end{table}
% %\texttt{fontsize}: \showFontSize

% Jeśli w pracy zostaną użyte otoczenia matematyczne, to w dokumencie wynikowym pojawią się dodatkowe czcionki (domyślne latexowe czcionki do wyrażeń matematycznych). Dzięki zastosowaniu opcji \texttt{extrafontsizes} w klasie \texttt{memoir} nie dość, że otrzymuje się większe czcionki (30pt), to jeszcze zamiast \texttt{Computer Modern} do wzorów matematycznych jest stosowana czcionka \texttt{Latin Modern} (wywodząca się z \texttt{Computer Modern}).
% Stąd lista wszystkich użytych czcionek może być następująca:
% \begin{lstlisting}[basicstyle=\footnotesize\ttfamily]
% EBGaramond12-Regular
% GaramondNo8-Reg-Norml
% TeXGyreTermes-Regular-Normalna
% TeXGyreTermes-Bold-Pogrubiona
% TeXGyreTermes-Italic-Normalna
% t1xtt-Nomal
% LMMathItalic12-Regular
% LMMathSymbols10-Regular
% LMMathExtension10-Regular
% LMRoman8-Regular
% \end{lstlisting}

% Aby wykorzystać te czcionki poza systemem LaTeX, wystarczy pobrać je spod adresów (ważnych na dzień
% 1.04.2016): 
% \url{https://www.ctan.org/tex-archive/fonts/cm/ps-type1/bakoma/ttf/?lang=en}, \url{http://www.gust.org.pl/projects/e-foundry/latin-modern}, \url{http://www.gust.org.pl/projects/e-foundry/tex-gyre}, \url{https://bitbucket.org/georgd/eb-garamond/downloads},
% a następnie zainstalować w systemie. Dzięki temu można będzie np.~edytować rysunki używając dokładnie tej samej czcionki, co czcionka użyta w dokumencie.

% \section{Formatowanie bloków tekstu}
% Każdy rozdział pracy powinien rozpoczynać się od nowej strony. Jej wygląd powinien być kontrolowany parametrami pokazanymi na rysunku~\ref{fig:LayChap}.
% \begin{figure}[t]
% \setlayoutscale{0.6}
% \centering
% \chapterdiagram
% \caption{Parametry sterujące wielkościami odstępów na stronie z tytułem rozdziału} 
% \label{fig:LayChap}
% \end{figure}
% W niniejszym szablonie (dokument klasy \texttt{memoir} z opcją \texttt{[12pt]}) przyjęto następujące wartości tych parametrów:
% \begin{itemize}
% \item \verb?\beforechapskip? (\printlength{\beforechapskip}) + \verb?\baselineskip? of \verb+\huge+ (30pt) + \verb+\topskip+ (\printlength{\topskip}) = 92pt (3.246cm)
% \item \verb?\midchapskip? (\printlength{\midchapskip}) + \verb?\baselineskip? of \verb+\Huge+ (37pt) = 57 pt (2.011cm)
% \item \verb?\afterchapskip? (\printlength{\afterchapskip}) + \verb+\baselineskip+ of \verb+\normalsize+ (14.5pt) = 54.5pt (1.923cm)
% \end{itemize}

% Nieco kłopotów może sprawić dobre ustawienie na stronie tytułów nienumerowanych rozdziałów oraz list generowanych automatycznie (Skróty, Spis treści, Spis rysunków, Spis tabel, Indeks rzeczowy). W szablonie w tym celu zdefiniowano nowy styl rozdziału komendami jak niżej (w szablonie są to komendy zamarkowane)
% \begin{lstlisting}[basicstyle=\footnotesize\ttfamily]
% \newlength{\linespace}
% \setlength{\linespace}{-\beforechapskip-\topskip+\headheight+\topsep}
% \makechapterstyle{noNumbered}{%
% \renewcommand\chapterheadstart{\vspace*{\linespace}}
% }
% \end{lstlisting}
% oraz dokonano przełączenia stylów rozdziałów komendami \verb?\chapterstyle{nonumbered}? oraz \verb?\chapterstyle{default}? podczas dołączania do dokumentu wymienionych nienumerowanych rozdziałów i list. Aby ,,podnieść do góry'' tytuły nienumerowanych rozdziałów (gdy jest to rzeczywiście konieczne) wystarczy odmarkować wspomniane komendy.

% Tytuły rozdziałów, sekcji, podsekcji itd.\ nie powinny kończyć się kropką. Odległości pomiędzy tekstem wiodącym a tytułem sekcji powinien być regulowany parametrami pokazanymi na rysunku~\ref{fig:LaySec}. 
% \begin{figure}[t]
% %\setlayoutscale{1}
% \runinheadfalse
% \drawparameterstrue
% \drawheading{\Large\bfseries}
% \caption{Kontrola ustawień odległości w tytułach kolejnych sekcji} 
% \label{fig:LaySec}
% \end{figure}
% Rozmiar \verb?\baselineskip? zależy od rozmiaru czcionki (zobacz tabela~\ref{tab:secfonts}), zaś \texttt{beforeskip} i \texttt{secskip} od poziomu sekcji. W niniejszym szablonie przyjęto następujące wartości tych parametrów (są to wartości dobierane elastycznie podczas kompilacji):
% \begin{itemize}
% \item \texttt{indent = 14.5pt}
% \item \texttt{parskip = \printlength{\parskip}}
% \item \texttt{beforesecskip = \printlength{\beforesecskip}}
% \item \texttt{aftersecskip = \printlength{\aftersecskip}}
% \item \texttt{beforesubsecskip = \printlength{\beforesubsecskip}}
% \item \texttt{aftersubsecskip = \printlength{\aftersubsecskip}}
% \item \texttt{beforesubsubsecskip = \printlength{\beforesubsecskip}}
% \item \texttt{aftersubsubsecskip = \printlength{\aftersubsecskip}}
% \end{itemize}

% W szablonie obowiązują również następujące wartości parametrów odpowiedzialnych za odstępy pomiędzy pływającymi figurami, tekstami oraz tekstem i figurą:
% \begin{itemize}
% \item \texttt{floatsep = \printlength{\floatsep}}
% \item \texttt{intextsep = \printlength{\intextsep}}
% \item \texttt{textfloatsep = \printlength{\textfloatsep}}
% \end{itemize}

% Pierwsza linia pierwszego akapitu w bloku (po tytule rozdziału, sekcji, podsekcji, podpodsekcji) nie może mieć wcięcia. Pierwsze linie w kolejnych akapitach już powinny mieć wcięcie równe \texttt{14.5pt}. Tekst w akapitach powinien być wyrównany z obu stron. 


% Strony powinny być numerowane numeracją ciągłą (sekwencja arabskich cyfr). Numery stron powinny być umieszczone w ich stopkach (tj.\ tak jak w niniejszym dokumencie). Wyjątkiem są tutaj pierwsze strony rozdziałów oraz strona tytułowa -- na nich numery nie powinny się pojawić.

% \section{Opisy tabel i rysunków}
% Podpisy powinny być umieszczane pod rysunkami lub nad tabelami wraz z etykietą składającą się ze skrótu Rys.\ lub Tab.\ oraz numeru. Podpisy te nie powinny mieć końcowej kropki. Numery występujący w podpisach powinny zaczynać się numerem rozdziału, po którym następuje kolejny numer rysunku lub tabeli w obrębie rozdziału. Etykieta powinna kończyć się dwukropkiem, po którym następuje tekst podpisu. Numer rozdziału powinien być rozdzielony kropką od kolejnego numeru w rysunku bądź tabeli w rozdziale (liczniki tabel i rysunków są rozłączne). Należy pamiętać o tym, żeby w całej pracy tabele miały podobny wygląd (rodzaj czcionki, ewentualne pogrubienia w nagłówku itp.). %Źródła należy podawać pod tabelą.

% \section{Przypisy dolne}
% Istnieje możliwość zamieszczania przypisów na dole strony, choć nie jest to zalecane (przykładowo~\footnote{Tekst przypisu}). Sposób parametryzowania ich wyglądu pokazano na rysunku~\ref{fig:fp}. W szablonie wykorzystano następujące, domyślne wartości tych parametrów:
% \begin{lstlisting}[basicstyle=\footnotesize\ttfamily]
% \footins = 12pt \footnotesep = 8pt
% \baselineskip = 10pt note separation = 40pt
% rule thickness = 0.4pt
% rule length = 0.25 times the \textwidth
% \end{lstlisting}
% \begin{figure}[htb]
% \setlayoutscale{0.3}
% \drawfootnote
% \caption{Parametry sterujące przypisami dolnymi} \label{fig:fp}
% \end{figure}
% %\tryfootins
% %\tryfootnotesep
% %\tryfootnotebaseline
% %\tryfootruleheight
% %\tryfootrulefrac

% %\footins = 12pt \footnotesep = 8pt
% %\baselineskip = 10pt note separation = 40pt
% %rule thickness = 0.4pt
% %rule length = 0.25 times the \textwidth

% \section{Formatowanie spisu treści}
% W klasie \texttt{memoir} istnieją komendy pozwalające dość dobrze zarządzać wyglądem spisu treści. Na rysunku~\ref{fig:ltoc} pokazano, za pomocą jakich parametrów można wpływać na finalną jego postać. W szablonie wykorzystano następujące, domyślne ich wartości:
% \begin{lstlisting}[basicstyle=\footnotesize\ttfamily]
% indent = 18pt 
% numwidth = 28pt
% \@tocrmarg = 31pt 
% \@pnumwidth = 19pt
% \@dotsep = 4.5
% \end{lstlisting}

% \begin{figure}[h]
% \setlayoutscale{0.5}
% \drawtoc
% \caption{Parametryzacja wyglądu spisu treści} \label{fig:ltoc}
% \end{figure}

% %\begin{figure}
% %\setlayoutscale{0.8}
% %\currenttoc
% %\drawparametersfalse
% %\drawtoc
% %\caption{Parametry definiujące postać spisu treści w niniejszym szablonie} \label{fig:thistoc}
% %\end{figure}

% \section{Formatowanie list wyliczeniowych i wypunktowań}
% Standardowo sposób formatowania list można parametryzować jak pokazano na rysunku~\ref{fig:listlay}. Jednak czasem trudno poradzić sobie z niektórymi rzeczami, jak np.~znakami wypunktowania. Dlatego w szablonie wykorzystano pakiet \texttt{enumi}. Pozwala on na łatwe zarządzanie wyglądem list. W szablonie zastosowano następujące globalne ustawienia dla tego pakietu:
% \begin{lstlisting}[basicstyle=\footnotesize\ttfamily]
% \usepackage{enumitem} 
% \setlist{noitemsep,topsep=4pt,parsep=0pt,partopsep=4pt,leftmargin=*} 
% \setenumerate{labelindent=0pt,itemindent=0pt,leftmargin=!,label=\arabic*.} 
% \setlistdepth{4} 
% \setlist[itemize,1]{label=$\bullet$} 
% \setlist[itemize,2]{label=\normalfont\bfseries\textendash}
% \setlist[itemize,3]{label=$\ast$}
% \setlist[itemize,4]{label=$\cdot$}
% \renewlist{itemize}{itemize}{4}
% \end{lstlisting}
% \begin{figure}[h]
% \centering
% \setlayoutscale{0.4}
% \drawparameterstrue
% \drawlist
% \caption{Parametryzacja list wyliczeniowych i wypunktowań}\label{fig:listlay}
% \end{figure}

% W~podrozdziale~\ref{sec:Styl} pokazano przykład wykorzystania możliwości komend oferowanych w~pakiecie \texttt{enumi}.

% \section{Wzory matematyczne}
% Wzory matematyczne, jeśli mają być osobnymi formułami, powinny być wycentrowane, z~numeracją umieszczoną na końcu linii i ujętą w okrągłe nawiasy (zobacz równanie (\ref{eq:xdx})). Numery równań powinny zawierać numer rozdziału oraz kolejny numer równania w obrębie rozdziału (podobnie jak przy numerowaniu rysunków i tabel). Spełnienie tych warunków zapewnia otoczenie \verb?equation?. Nie wszystkie formuły trzeba numerować (nienumerowane wzory można osiągnąć stosując otoczenie \verb?\equation*?). Właściwie należy numerować tylko te, do których tworzy się jakieś odniesienia w tekście. Jeśli wzory umieszczane są w linijce tekstu, to można zastosować otoczenie matematyczne inline, jak w~przykładzie $\int_{0}^{10\nu\sum i}{x dx}$ (wyprodukowanym komendą \verb?$\int_{0}^{10\nu\sum i}{x dx}$?). Tylko że wtedy może dojść do rozszerzenia odstępów pomiędzy liniami tekstu (aby zmieścił się wzór).
% \begin{equation}\label{eq:xdx}
% \int_{0}^{10\nu\sum i}{x dx}
% \end{equation}
