\pdfbookmark[0]{Skróty}{skroty.1}% 
%%\phantomsection
%%\addcontentsline{toc}{chapter}{Skróty}
\chapter*{Skróty}
\label{sec:skroty}
\noindent\vspace{-\topsep-\partopsep-\parsep} % Jeśli zaczyna się od otoczenia description, to otoczenie to ląduje lekko niżej niż wylądowałby zwykły tekst, dlatego wstawiano przesunięcie w pionie
\begin{description}[labelwidth=*]
  \item [DAW] (ang.\ \emph{Digital Audio Workstation}) %-- jednostka arytmetyczno--logiczna
  \item [NN] (ang.\ \emph{Neural Network}) %-- jednostka arytmetyczno--logiczna
  \item [FFT] (ang.\ \emph{Fast Fourier Transform}) %-- jednostka arytmetyczno--logiczna
  \item [STFT] (ang.\ \emph{Short-time Fourier Transform}) %-- jednostka arytmetyczno--logiczna
\end{description}
