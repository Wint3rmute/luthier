\pdfbookmark[0]{Skróty}{skroty.1}% 
%%\phantomsection
%%\addcontentsline{toc}{chapter}{Skróty}
\chapter*{Skróty}
\label{sec:skroty}
\noindent\vspace{-\topsep-\partopsep-\parsep} % Jeśli zaczyna się od otoczenia description, to otoczenie to ląduje lekko niżej niż wylądowałby zwykły tekst, dlatego wstawiano przesunięcie w pionie
\begin{description}[labelwidth=*]
  \item [OGC] (ang.\ \emph{Open Geospatial Consortium}) %-- jednostka arytmetyczno--logiczna
  \item [XML] (ang.\ \emph{eXtensible Markup Language})
  \item [SOAP] (ang.\ \emph{Simple Object Access Protocol})
  \item [WSDL] (ang.\ \emph{Web Services Description Language})
  \item [UDDI] (ang.\ \emph{Universal Description Discovery and Integration})
  \item [GIS] (ang.\ \emph{Geographical Information System})
  \item [SDI] (ang.\ \emph{Spatial Data Infrastructure})
  \item [ISO] (ang.\ \emph{International Standards Organization})
  \item [WMS] (ang.\ \emph{Web Map Service})
  \item [WFS] (ang.\ \emph{Web Feature Service})
  \item [WPS] (ang.\ \emph{Web Processing Service})
  \item [GML] (ang.\ \emph{Geography Markup Language})
  \item [SRG] (ang.\ \emph{Seeded Region Growing})
  \item [SOA] (ang.\ \emph{Service Oriented Architecture })
  \item [IT] (ang.\ \emph{Information Technology })
\end{description}
